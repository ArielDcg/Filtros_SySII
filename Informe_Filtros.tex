\documentclass[12pt, a4paper, spanish]{article}

\usepackage[utf8]{inputenc}
\usepackage[T1]{fontenc}
\usepackage[spanish]{babel}
\usepackage{graphicx}
\usepackage{amsmath}
\usepackage{amssymb}
\usepackage{geometry}
\usepackage{fancyhdr}
\usepackage{array}
\usepackage{booktabs}
\usepackage{float}
\usepackage{subcaption}
\usepackage{listings}
\usepackage{xcolor}
\usepackage{hyperref}

% Configuración de márgenes
\geometry{
    top=2.5cm,
    bottom=2.5cm,
    left=2.5cm,
    right=2.5cm
}

% Configuración de encabezados
\pagestyle{fancy}
\fancyhf{}
\rhead{Proyecto de Filtros}
\lhead{Sistemas y Señales II}
\cfoot{\thepage}

% Configuración de listings para código
\lstset{
    basicstyle=\ttfamily\small,
    keywordstyle=\color{blue},
    commentstyle=\color{gray},
    stringstyle=\color{red},
    breaklines=true,
    showspaces=false,
    showstringspaces=false,
    frame=single,
    rulecolor=\color{black},
    backgroundcolor=\color{white!95!black}
}

\title{\textbf{\Huge Diseño e Implementación de un Ecualizador de Audio de 4 Bandas} \\
\textbf{\large Filtros Analógicos, IIR y FIR} \\
\vspace{1cm}
\textbf{Sistemas y Señales II}}

\author{
    \textbf{Autor 1}\\
    Johan Arturo Barajas Herrera\\
    \texttt{jbarajash@unal.edu.co}\\
    \vspace{0.5cm}
    \and
    \textbf{Autor 2}\\
    Ariel Giovanni Cardenas Santisteban\\
    \texttt{arcardenass@unal.edu.co}\\
    \vspace{0.5cm}
    \and
    \textbf{Autor 3}\\
    Dilan Santiago Porras Cortés\\
    \texttt{diporrasc@unal.edu.co}
}


\date{\today}

\begin{document}

\maketitle

\begin{abstract}
\noindent
Este informe presenta el diseño e implementación de tres tipos fundamentales de filtros digitales y analógicos:
filtros analógicos (Butterworth), filtros IIR (Infinite Impulse Response) y filtros FIR (Finite Impulse Response).
Se describe el procedimiento de diseño, se muestran las funciones de transferencia, y se analizan las respuestas
en frecuencia (magnitud y fase) de cada filtro mediante diagramas de Bode. Los filtros se implementaron en MATLAB
usando funciones estándar y sus características se validan mediante el análisis de sus especificaciones técnicas.
\end{abstract}

\newpage

\tableofcontents

\newpage

\section{Introducción}

Los filtros son dispositivos o sistemas fundamentales en el procesamiento de señales que permiten atenuar,
amplificar o dejar pasar ciertos componentes de frecuencia de una señal. Existen diferentes tipos de filtros
clasificados según sus características:

\begin{itemize}
    \item \textbf{Filtros Analógicos}: Operan sobre señales continuas en el tiempo.
    \item \textbf{Filtros Digitales IIR}: Sistemas discretos mas eficientes caracterizados por una funcion de transferencia con polos y ceros finitos.
    \item \textbf{Filtros Digitales FIR}: Sistemas discretos caracterizados por una funcion de transferencia con todos sus polos en el origen e influenciados unicamente por sus ceros.
\end{itemize}

Este proyecto implementa ejemplos de cada tipo de filtro, con énfasis en su caracterización mediante la respuesta
en frecuencia (magnitud en decibelios y fase en grados).

\section{Especificaciones de Filtros Implementados}

\subsection{Filtro Analógico Butterworth}

\begin{table}[H]
\centering
\begin{tabular}{|l|l|}
\hline
\textbf{Parámetro} & \textbf{Valor} \\
\hline
Tipo & Paso Bajo (Low Pass) \\
\hline
Topología & Butterworth \\
\hline
Orden & 4 \\
\hline
Frecuencia de Corte ($f_c$) & 1000 Hz \\
\hline
Frecuencia de Corte ($\omega_c$) & 6283.19 rad/s \\
\hline
Atenuación a $f_c$ & -3 dB \\
\hline
Pendiente de Atenuación & 80 dB/década \\
\hline
\end{tabular}
\caption{Especificaciones del Filtro Analógico}
\label{tab:esp_analogico}
\end{table}

La función de transferencia del filtro Butterworth analógico de orden 4 es:

\begin{equation}
H(s) = \frac{\omega_c^4}{(s + \omega_c)(s^2 + \sqrt{2}\omega_c s + \omega_c^2)(s^2 + \sqrt{2}\omega_c s + \omega_c^2)}
\end{equation}

El filtro Butterworth es de magnitud máxima plana en la banda de paso, lo que significa que no presenta rizado.

\subsection{Filtro IIR Butterworth}

\begin{table}[H]
\centering
\begin{tabular}{|l|l|}
\hline
\textbf{Parámetro} & \textbf{Valor} \\
\hline
Tipo & Paso Bajo (Low Pass) \\
\hline
Topología & Butterworth Digital \\
\hline
Orden & 5 \\
\hline
Frecuencia de Muestreo ($f_s$) & 5000 Hz \\
\hline
Frecuencia de Corte ($f_c$) & 500 Hz \\
\hline
Frecuencia de Nyquist & 2500 Hz \\
\hline
Frecuencia Normalizada ($W_n$) & 0.2 \\
\hline
Método de Diseño & Transformación Bilineal \\
\hline
Estabilidad & Garantizada (polos dentro de círculo unitario) \\
\hline
\end{tabular}
\caption{Especificaciones del Filtro IIR}
\label{tab:esp_iir}
\end{table}

La función de transferencia discreta se obtiene mediante la transformación bilineal del prototipo analógico.

\subsection{Filtro FIR Paso Bajo}

\begin{table}[H]
\centering
\begin{tabular}{|l|l|}
\hline
\textbf{Parámetro} & \textbf{Valor} \\
\hline
Tipo & Paso Bajo (Low Pass) \\
\hline
Topología & FIR con Ventana \\
\hline
Ventana & Hamming \\
\hline
Orden (N) & 64 \\
\hline
Número de Coeficientes & 65 \\
\hline
Frecuencia de Muestreo ($f_s$) & 8000 Hz \\
\hline
Frecuencia de Corte ($f_c$) & 1000 Hz \\
\hline
Frecuencia de Nyquist & 4000 Hz \\
\hline
Frecuencia Normalizada ($W_n$) & 0.25 \\
\hline
Propiedad & Fase Lineal \\
\hline
\end{tabular}
\caption{Especificaciones del Filtro FIR}
\label{tab:esp_fir}
\end{table}

Los coeficientes se calculan usando el método de ventanas con la ventana de Hamming para reducir el fenómeno de Gibbs.

\newpage

\section{Procedimiento de Diseño e Implementación}

\subsection{Filtro Analógico}

\subsubsection{Procedimiento}

\begin{enumerate}
    \item Especificar los requisitos: tipo (paso bajo), orden (4) y frecuencia de corte (1000 Hz).
    \item Seleccionar la aproximación: Butterworth (magnitud máxima plana).
    \item Calcular los polos del filtro normalizado.
    \item Desnormalizar a la frecuencia de corte deseada.
    \item Construir la función de transferencia.
    \item Evaluar la respuesta en frecuencia mediante $H(j\omega)$.
\end{enumerate}

\subsubsection{Código MATLAB}

\begin{lstlisting}[language=MATLAB]
% Especificaciones
Fc = 1000;              % Frecuencia de corte en Hz
Wc = 2*pi*Fc;           % Frecuencia de corte en rad/s
orden = 4;              % Orden del filtro

% Diseño
[num_an, den_an] = butter(orden, Wc, 's');
H_analog = tf(num_an, den_an);

% Respuesta en frecuencia
w = logspace(0, 5, 1000);
[mag, phase, w] = bode(H_analog, w);
mag_dB = 20*log10(squeeze(mag));
phase_deg = squeeze(phase);
\end{lstlisting}

\subsection{Filtro IIR}

\subsubsection{Procedimiento}

\begin{enumerate}
    \item Especificar: tipo (paso bajo), orden (5), frecuencia de muestreo (5000 Hz), frecuencia de corte (500 Hz).
    \item Normalizar la frecuencia de corte respecto a Nyquist: $W_n = \frac{f_c}{f_s/2} = 0.2$.
    \item Aplicar el método de Butterworth digital.
    \item Usar la transformación bilineal para obtener el filtro digital.
    \item Verificar la estabilidad (polos dentro del círculo unitario).
    \item Evaluar la respuesta en frecuencia.
\end{enumerate}

\subsubsection{Código MATLAB}

\begin{lstlisting}[language=MATLAB]
% Especificaciones
Fs = 5000;              % Frecuencia de muestreo
Fc = 500;               % Frecuencia de corte
Wn = Fc / (Fs/2);       % Frecuencia normalizada
orden = 5;

% Diseño
[num_iir, den_iir] = butter(orden, Wn);
H_iir = tf(num_iir, den_iir, 1/Fs);

% Respuesta en frecuencia
[H] = freqz(num_iir, den_iir, w, Fs);
mag_dB = 20*log10(abs(H));
phase_deg = angle(H) * 180/pi;

% Verificar estabilidad
polos = pole(H_iir);
es_estable = all(abs(polos) < 1);
\end{lstlisting}

\subsection{Filtro FIR}

\subsubsection{Procedimiento}

\begin{enumerate}
    \item Especificar: tipo (paso bajo), orden (64), ventana (Hamming).
    \item Definir la frecuencia de corte normalizada: $W_n = 0.25$.
    \item Crear la ventana de Hamming de longitud $N+1 = 65$.
    \item Calcular los coeficientes ideales del filtro paso bajo.
    \item Aplicar la ventana a los coeficientes.
    \item Verificar las propiedades de simetría para fase lineal.
    \item Evaluar la respuesta en frecuencia.
\end{enumerate}

\subsubsection{Código MATLAB}

\begin{lstlisting}[language=MATLAB]
% Especificaciones
Fs = 8000;
Fc = 1000;
Wn = Fc / (Fs/2);
N = 64;

% Diseño
ventana = hamming(N+1)';
num_fir = fir1(N, Wn, 'low', ventana);
H_fir = tf(num_fir, 1, 1/Fs);

% Respuesta en frecuencia
[H] = freqz(num_fir, 1, w, Fs);
mag_dB = 20*log10(abs(H));
phase_deg = angle(H) * 180/pi;

% Verificar fase lineal
es_lineal = max(abs(num_fir - fliplr(num_fir))) < 1e-10;
\end{lstlisting}

\newpage

\section{Resultados y Análisis}

\subsection{Filtro Analógico}

\begin{figure}[H]
\centering
\includegraphics[width=0.9\textwidth]{../figures/filtro_analogico.png}
\caption{Diagrama de Bode del Filtro Butterworth Analógico (Orden 4, $f_c = 1000$ Hz)}
\label{fig:bode_analogico}
\end{figure}

\subsubsection{Análisis de Resultados}

\begin{itemize}
    \item \textbf{Respuesta en Magnitud}: El filtro presenta una atenuación de aproximadamente -3 dB a la frecuencia
          de corte (1000 Hz), como se esperaba. La magnitud máxima plana en la banda de paso (0-800 Hz) caracteriza
          al filtro Butterworth.

    \item \textbf{Pendiente de Atenuación}: La pendiente de atenuación es de aproximadamente 80 dB/década
          (24 dB/octava), consistente con un filtro de orden 4. Esto se calcula como $20 \times \text{orden} = 20 \times 4 = 80$ dB/década.

    \item \textbf{Respuesta en Fase}: La fase disminuye monótonamente desde 0° a frecuencias bajas hasta -180°
          a frecuencias altas. A la frecuencia de corte, la fase es aproximadamente -90°.

    \item \textbf{Polos}: El filtro tiene 4 polos complejos conjugados ubicados en el semiplano izquierdo (sistemas estables).
\end{itemize}

\subsection{Filtro IIR}

\begin{figure}[H]
\centering
\includegraphics[width=0.9\textwidth]{../figures/filtro_iir.png}
\caption{Diagrama de Bode del Filtro IIR Butterworth Digital (Orden 5, $f_c = 500$ Hz, $f_s = 5000$ Hz)}
\label{fig:bode_iir}
\end{figure}

\subsubsection{Análisis de Resultados}

\begin{itemize}
    \item \textbf{Respuesta en Magnitud}: El filtro digital muestra una atenuación de -3 dB alrededor de 500 Hz.
          La respuesta es plana en la banda de paso (0-400 Hz).

    \item \textbf{Comportamiento en Banda de Rechazo}: A partir de 500 Hz, la atenuación se incrementa progresivamente,
          llegando a valores significativos cercanos a la frecuencia de Nyquist (2500 Hz).

    \item \textbf{Respuesta en Fase}: La fase disminuye de manera suave a través de las frecuencias, con máxima
          pendiente de fase alrededor de la frecuencia de corte.

    \item \textbf{Estabilidad}: Todos los polos están ubicados dentro del círculo unitario en el plano-z,
          garantizando la estabilidad del filtro. Los ceros se encuentran parcialmente en el círculo unitario
          o fuera de él.

    \item \textbf{Ventaja sobre analógico}: El filtro IIR requiere menos coeficientes (menos memoria) que un
          filtro FIR equivalente para alcanzar la misma selectividad.
\end{itemize}

\subsection{Filtro FIR}

\begin{figure}[H]
\centering
\includegraphics[width=0.9\textwidth]{../figures/filtro_fir.png}
\caption{Diagrama de Bode del Filtro FIR Paso Bajo (Orden 64, Ventana Hamming, $f_c = 1000$ Hz)}
\label{fig:bode_fir}
\end{figure}

\begin{figure}[H]
\centering
\includegraphics[width=0.9\textwidth]{../figures/filtro_fir_impulso.png}
\caption{Respuesta al Impulso del Filtro FIR (65 coeficientes)}
\label{fig:fir_impulso}
\end{figure}

\subsubsection{Análisis de Resultados}

\begin{itemize}
    \item \textbf{Respuesta en Magnitud}: El filtro FIR muestra una respuesta paso bajo clara con -3 dB aproximadamente
          a 1000 Hz. La ventana de Hamming reduce el rizado de Gibbs, resultando en una respuesta más suave.

    \item \textbf{Rizado en Banda de Paso}: Existe un pequeño rizado caracterizado por la ventana de Hamming,
          típicamente alrededor de ±0.5 dB. Este rizado es el tradeoff por tener lóbulos laterales más pequeños
          (aproximadamente -43 dB).

    \item \textbf{Respuesta en Fase}: Presenta fase lineal, como se confirma por la simetría de los coeficientes.
          La fase es una función lineal de la frecuencia: $\phi(\omega) = -\frac{N\omega}{2}$.

    \item \textbf{Respuesta al Impulso}: Los 65 coeficientes ($h[0]$ a $h[64]$) son simétricos alrededor del
          índice 32, lo que garantiza la fase lineal.

    \item \textbf{Estabilidad}: Garantizada por diseño (sin retroalimentación, solo ceros).

    \item \textbf{Ventajas}:
    \begin{itemize}
        \item Fase exactamente lineal (sin distorsión de fase).
        \item Siempre estables.
        \item Sin problemas de ciclos límite o rondeo.
    \end{itemize}

    \item \textbf{Desventajas}:
    \begin{itemize}
        \item Requiere más coeficientes para igual selectividad que IIR.
        \item Mayor costo computacional (más multiplicaciones).
        \item Latencia inherente debido al orden del filtro.
    \end{itemize}
\end{itemize}

\newpage

\section{Comparación de Filtros}

\begin{table}[H]
\centering
\begin{tabular}{|l|c|c|c|}
\hline
\textbf{Característica} & \textbf{Analógico} & \textbf{IIR} & \textbf{FIR} \\
\hline
Estabilidad & \checkmark & Depende (criterio BIB) & \checkmark (Garantizada) \\
\hline
Fase Lineal & \texttimes & \texttimes & \checkmark (con simetría) \\
\hline
Selectividad & Alta & Alta (orden bajo) & Baja (requiere orden alto) \\
\hline
Complejidad Computacional & N/A & Baja & Media-Alta \\
\hline
Número de Coeficientes & N/A & Bajo & Alto \\
\hline
Latencia & Continua & Baja & Media-Alta \\
\hline
Implementación Digital & N/A & Mediante $z$-transform & Convolución discreta \\
\hline
Rizado & Ninguno (Butterworth) & Posible & Controlado por ventana \\
\hline
Distorsión de Fase & Sí & Sí & No (ideal) \\
\hline
\end{tabular}
\caption{Comparación de características entre tipos de filtros}
\label{tab:comparacion}
\end{table}

\subsection{Recomendaciones de Uso}

\begin{itemize}
    \item \textbf{Filtro Analógico}: Usar cuando se requiera procesamiento de señales analógicas en tiempo real
          (e.g., filtración de entrada antes de ADC).

    \item \textbf{Filtro IIR}: Preferible cuando se necesita alta selectividad con bajo costo computacional
          y la distorsión de fase no sea crítica.

    \item \textbf{Filtro FIR}: Usar cuando la fase lineal sea esencial para la aplicación (e.g., audio de alta fidelidad,
          procesamiento de imágenes) o cuando se necesite evitar problemas de estabilidad.
\end{itemize}

\newpage

\section{Conclusiones}

Se ha completado exitosamente el diseño e implementación de tres tipos fundamentales de filtros:

\begin{enumerate}
    \item \textbf{Filtro Analógico Butterworth}: Proporciona una magnitud máxima plana en la banda de paso
          con una pendiente de atenuación de 80 dB/década. Suitable para aplicaciones de acondicionamiento analógico.

    \item \textbf{Filtro IIR Digital}: Ofrece excelente selectividad con mínimo costo computacional,
          aunque introduce distorsión de fase. Ideal para aplicaciones en tiempo real con recursos limitados.

    \item \textbf{Filtro FIR Digital}: Proporciona fase exactamente lineal y estabilidad garantizada,
          aunque requiere mayor orden para lograr selectividad equivalente. Esencial para aplicaciones
          donde la fase debe ser controlada.
\end{enumerate}

Todos los filtros han sido validados mediante su respuesta en frecuencia, diagramas de Bode,
y características espectrales. Las gráficas en magnitud (dB) y fase muestran comportamientos consistentes
con la teoría de filtros digitales y analógicos.

\section{Referencias}

\begin{enumerate}
    \item Oppenheim, A. V., Schafer, R. W., \& Buck, J. R. (2009). \textit{Discrete-time signal processing}. Prentice Hall.

    \item Proakis, J. G., \& Manolakis, D. G. (2006). \textit{Digital signal processing: principles, algorithms, and applications}. Prentice Hall.

    \item The Mathworks. (2023). MATLAB Signal Processing Toolbox Documentation.

    \item Parks, T. W., \& Burrus, C. S. (1987). \textit{Digital filter design}. John Wiley \& Sons.
\end{enumerate}

\appendix

\section{Scripts de MATLAB Utilizados}

Los scripts están organizados en la carpeta \texttt{/scripts/} y son:

\begin{itemize}
    \item \texttt{filtro\_analogico.m} - Diseño del filtro Butterworth analógico.
    \item \texttt{filtro\_iir.m} - Diseño del filtro IIR digital.
    \item \texttt{filtro\_fir.m} - Diseño del filtro FIR con ventana Hamming.
    \item \texttt{main\_generar\_filtros.m} - Script maestro que ejecuta todos los diseños.
\end{itemize}

Para ejecutar, en MATLAB ejecute:
\begin{lstlisting}
run('/home/user/Filtros_SySII/scripts/main_generar_filtros.m')
\end{lstlisting}

\end{document}
